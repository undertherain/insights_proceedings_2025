
Publication of negative results is difficult in most fields, and the current focus on benchmark-driven per-
formance improvement exacerbates this situation and implicitly discourages hypothesis-driven research.
As a result, the development of NLP models often devolves into a product of tinkering and tweaking,
rather than science. Furthermore, it increases the time, effort, and carbon emissions spent on developing
and tuning models, as the researchers have little opportunity to learn from what has already been tried
and failed.

The mission of the workshop on Insights from Negative Results in NLP is to provide a venue for many
kinds of negative results, with the hope that they could yield useful insights and provide a much-needed
reality check on the successes of deep learning models in NLP. In particular, we solicit the following
types of contributions:
\begin{itemize}
    \item broadly applicable recommendations for training/fine-tuning, especially if X that didn’t work is
something that many practitioners would think reasonable to try, and if the demonstration of X’s
failure is accompanied by some explanation/hypothesis;
\item  ablation studies of components in previously proposed models, showing that their contributions
are different from what was initially reported;
\item  datasets or probing tasks showing that previous approaches do not generalize to other domains or
language phenomena;
\item  trivial baselines that work suspiciously well for a given task/dataset;
\item  cross-lingual studies showing that a technique X is only successful for a certain language or lan-
guage family;
\item  experiments on (in)stability of the previously published results due to hardware, random initiali-
zations, preprocessing pipeline components, etc;
\item  theoretical arguments and/or proofs for why X should not be expected to work;
\item  demonstration of issues with under-reporting of training details of pre-trained models, including
test data contamination and invalid comparisons.

\end{itemize}

The fifth iteration of the Workshop on Insights from Negative Results attracted 23 submissions and 2 from
ACL Rolling Reviews.
%In terms of topics/themes, 4 papers from our accepted proceedings discussed
%“zero-shot / few-shot learning / low-resource settings”; 1 discussed “cross-modal fine-tuning”; 6 papers
%examined pre-trained representations / generalization; 1 dealt with tokenization; 6 on the topic of “LLM
%Reasoning / Alignment / Evaluations / Probing”; 1 on Multi-task Learning. Some submissions fit in more
%than one category.
We accepted 16 papers, resulting in  64\% acceptance rate.
We hope the workshop will continue to contribute to the many reality-check discussions on progress in
NLP. If we do not talk about things that do not work, it is harder to see what the biggest problems are
and where the community effort is the most needed